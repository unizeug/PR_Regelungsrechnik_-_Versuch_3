\newcommand{\institut}{}
\newcommand{\fachgebiet}{Regelungstechnik}
\newcommand{\veranstaltung}{Praktikum Grundlagen der Regelungstechnik}
\newcommand{\pdfautor}{Dirk Barbendererde (321 836), Boris Henckell (325 779)}
\newcommand{\autor}{Dirk Barbendererde (321 836)\\ Boris Henckell (325 779)}
\newcommand{\pdftitle}{Praktikum\ Regelungstechnik\ Versuch\ 3}
\newcommand{\prototitle}{Praktikum Regelungstechnik \\ Versuch 3}
\newcommand{\aufgabe}{}

\newcommand{\gruppe}{Gruppe: G1 Di 12-14}
\newcommand{\betreuer}{Betreuer: Markus Valtin}



\input{../../packages/tu_header_9}
%\begin{document}

% \lstlistoflistings
% \definecolor{darkgray}{rgb}{0.95,0.95,0.95}
% \lstset{language=Scilab}
% \lstset{inputencoding=utf8}
% %\lstset{extendedchars=true} % Umlaute an der richtigen stelle und nicht am Anfang ausgeben
% \lstset{backgroundcolor=\color{darkgray}}
% \lstset{numbers=left, numberstyle=\tiny, stepnumber=1, numbersep=7pt, breaklines=true}
% \lstset{keywordstyle=\color{red}\bfseries\emph}
% \lstset{
% breaklines,
% numbers=left,
% frame=single,
% xleftmargin=-2cm,
% xrightmargin=-1.5cm
% }
% % enables UTF-8 in source code: (dirty, dirty hack)
% \lstset{literate=
%     %Deutsch
%     {ä}{{\"a}}1 {ö}{{\"o}}1 {ü}{{\"u}}1 {Ä}{{\"A}}1 {Ö}
%     {{\"O}}1 {Ü}{{\"U}}1 {ß}{\ss}1
%     %Türkisch
%     {â}{{\^{a}}}1 {Â}{{\^{A}}}1 {ç}{{\c{c}}}1 {Ç}{{\c{C}}}1 {ğ}{{\u{g}}}1 {Ğ}{{\u{G}}}1 {ı}{{\i}}1 {İ}{{\.{I}}}1 {ö}{{\"o}}1 {Ö}{{\"O}}1 {ş}{{\c{s}}}1
%     {Ş}{{\c{S}}}1 {ü}{{\"u}}1 {Ü}{{\"U}}1
%     %Polish
%     {ą}{{\k{a}}}1 {ć}{{\'c}}1 {ę}{{\k{e}}}1 {ł}{{\l{}}}1 {ń}{{\'n}}1 {ó}{{\'o}}1 {ś}{{\'s}}1 {ż}{{\.z}}1 {ź}{{\'z}}1 {Ą}{{\k{A}}}1 {Ć}{{\'C}}1
%     {Ę}{{\k{E}}}1 {Ł}{{\L{}}}1 {Ń}{{\'N}}1 {Ó}{{\'O}}1 {Ś}{{\'S}}1 {Ż}{{\.Z}}1 {Ź}{{\'Z}}1
%     %Spanish
%     {á}{{\'a}}1 {é}{{\'e}}1 {í}{{\'i}}1 {ó}{{\'o}}1 {ú}{{\'u}}1 {ñ}{{\~n}}1
% }

%     \lstinputlisting{./praktikum6.sce}



%---------------------------------------------------------------------
%---------------------------------------------------------------------
%---------------------------------------------------------------------

\section{Vorbereitungsaufgaben}
\begin{quote}
	\hspace{-2em}
	\subsection{Linearisierung}
    Aufgabe:\\
    Linearisieren Sie das nichtlineare Modell um die Ruhelage ($z, \varphi$) = ($0, 0$) und berechnen Sie die
    Transferfunktion von der Stellgröße $u$ zum Ausgang $\varphi$. Skizzieren Sie die Pol-/Nullstellenverteilung des
    Systems.\vspace{1em}
    
    dfgsdfgdsfg
    
    
\end{quote} %Ende Section Vorbereitungsaufgaben 

%--------------------------------------------------------------------
%--------------------------------------------------------------------





\section{Auswertung}
\begin{quote}

    
\end{quote} %Ende Section Ergebnisse



%--------------------------------------------------------------------
%--------------------------------------------------------------------

\section{Scilabcode}
\begin{quote}
\begin{quote}
%     \lstinputlisting[
%         caption={Scilab-script},
%         language=scilab,
%         label=lst:scilab]
%         {./Scilab/Pendel2a.sce}
        


\end{quote}

	
\end{quote} %Ende section

%--------------------------------------------------------------------
%--------------------------------------------------------------------    


%\begin{thebibliography}{999}
%\bibitem {Ueberschwingweite} Prof. Dr.-Ing. Raisch, Jörg; Dipl.-Ing. Hess, Anne-Katrin; Dipl.-Ing. Seel, Thomas:
%Grundlagen der Regelungstechnik - 4.Praktikum, S.5
%\bibitem {Ausregelzeit} Prof. Dr.-Ing. Raisch, Jörg; Dipl.-Ing. Hess, Anne-Katrin; Dipl.-Ing. Seel, Thomas:
%Grundlagen der Regelungstechnik - 4.Praktikum, S.5

%\usepackage{url}


%\bibitem{krachler}Christian Krachler:
%\href{http://www.krachler.com/fileadmin/user_upload/arbeiten/Reglersynthese_Christian_Krachler.pdf}{Reglersynthese nach
% dem Frequenzkennlinienverfahren}, S16, S22, 08.05.2012

%http://krachler.com/fileadmin/user\_upload/arbeiten/Reglersynthese\_Christian\_Krachler.pdf


%Name, Vorname.; evtl. Name2, Vorname2.: Titel des Dokumentes
%oder Buches, Zeitschrift/Verlag/URL (Auflage, Erscheinungsort, -jahr), ggf. Seitenzahlen
%\bibitem [Wiki10] {DigitaleMesskette2} \url{www.wikipedia.org}, Zugriff 22.03.2010
%\end{thebibliography}


\end{document}
