\newcommand{\institut}{}
\newcommand{\fachgebiet}{Regelungstechnik}
\newcommand{\veranstaltung}{Praktikum Grundlagen der Regelungstechnik}
\newcommand{\pdfautor}{Dirk Barbendererde (321 836), Boris Henckell (325 779)}
\newcommand{\autor}{Dirk Barbendererde (321 836)\\ Boris Henckell (325 779)}
\newcommand{\pdftitle}{Praktikum\ Regelungstechnik\ Versuch\ 3}
\newcommand{\prototitle}{Praktikum Regelungstechnik \\ Versuch 3}
\newcommand{\aufgabe}{}

\newcommand{\gruppe}{Gruppe: G1 Di 12-14}
\newcommand{\betreuer}{Betreuer: Markus Valtin}



\input{../../packages/tu_header_9}
%\begin{document}


%     \lstinputlisting{./praktikum6.sce}



%---------------------------------------------------------------------
%---------------------------------------------------------------------
%---------------------------------------------------------------------

\section{Reglerentwurf}
\begin{quote}
	\hspace{-2em}
	\subsection{PID-Regler}
	\label{aufg:3.1}
	
	Aufgabe:\\
	Entwerfen Sie für das totzeitfreie System
	
	\begin{equation*}
    	\begin{split}
    		\tilde{G}(s) = \frac{V_0}{(s-s_{\infty 1})(s-s_{\infty 2})}
    	\end{split}
    \end{equation*}
	
	einen (realen) PID-Regler $K_{PID}$, indem sie beide Polstellen der Strecke kürzen. Die übrigen Parameter sollen so
	bestimmt werden, dass die Dynamik des Führungsverhaltens des resultierenden Regelkreises mit dem des folgenden Polpaares übereinstimmt:
	
	\begin{equation*}
    	\begin{split}
    		P(s) = \frac{1}{s^2 \omega^2 + s \frac{2 d}{\omega} + 1}
    	\end{split}
    \end{equation*}
	
	Der Regelkreis soll also relativ schnell sein, jedoch ohne dass dabei Überschwingen auftritt.\vspace{1em}
	
    \begin{quote}
        
        Da wir für ein nichtlineares System mit Totzeit keinen Regler entwerfen können entwerfen wir zunächst einen
        Regler für das folgende totzeitfreie System:
        
        \begin{equation*}
            \begin{split}
                \tilde{G}(s) = \frac{V_0}{(s-s_{\infty 1})(s-s_{\infty 2})}
            \end{split}
        \end{equation*}
            
        Die Dynamik des Führungsverhaltens des resultierenden Regelkreises soll mit dem des folgenden Polpaares
        ubereinstimmt:
        
        \begin{equation*}
            \begin{split}
                P(s) = \frac{1}{s^2 \omega^2 + s \frac{2 d}{\omega} + 1}
            \end{split}
        \end{equation*}
        
        
    \end{quote}
    
    
    \subsection{Pad\'e-Approximation}
    
    Aufgabe:\\
    Approximieren Sie den Term $e^{-sT_d}$, welcher für die Totzeit verantwortlich ist, mithilfe einer
    Pad\'e-Approximation erster Ordnung durch eine gebrochen-rationale Funktion und stellen Sie das approximierte
    Streckenmodell $\tilde{G}$ auf, indem Sie die Totzeit in $G$ durch ihre Approximation ersetzen. \vspace{1em}
    
    \begin{quote}
        
        Als nächstes haben wir die Totzeit mit einer Pad\'e-Approximation erster Ordnung durch eine gebrochen-rationale
        Funktion ersetzt und multiplizieren sie an unsere Strecke. Für diese Näherung des Systems ($\hat{G}$) entwerfen
        wir nun einen Regler.
        
        
    \end{quote}
    
    \subsection{Reglerentwurf für die Strecke $\hat{G}$}
    Aufgabe:\\
    Es soll ein Regler für das approximierte Streckenmodell $\tilde{G}$ mittels Polvorgabe entworfen werden, der
    folgende Eigenschaften des geschlossenen Kreises ermöglicht:
        
        \begin{itemize}
            
            \item kein Überschwingen der Regelgröße bei sprungförmigen Führungs- oder Störsignalen
            
            \item ungefähr gleiche Anstiegszeit der Regelgröße wie bei der Verwendung des PID-Regler aus
            \ref{aufg:3.1}
            
            \item Regelfehler $\to$ 0 für t $\to$ $\infty$ unter sprungförmigen Referenzen
        
        \end{itemize}
        \vspace{1em}
        
        \begin{enumerate}
            
            \item Stellen Sie den Regleransatz mit kleinstmöglicher Nennerordnung und Integratoranteil auf. Wie viele Pole
            müssen Sie vorgeben?\\
            
            \item Stellen Sie das Sollpolpolynom auf; verwenden sie hierzu die Pole des Polpaares aus Aufgabe \ref{aufg:3.1}
            und die Pole der Strecke $G$, wählen sie einen weiteren Pol bei ($s_\infty = -2$).\\
            
            \item Stellen Sie die Sylvester Matrix durch Koeffizientenvergleich des Polpolynoms des geschlossenen Kreises
            mit Ihrem Sollpolpolynom auf und berechnen sie die Reglerparameter mithilfe von Scilab.\\
        
        \end{enumerate}
        
        Mit dem Regleransatz\\
        
        \begin{equation*}
            \begin{split}
                K(s) = \frac{\beta_0 + \beta_1 s + \beta_2 s^2 + \beta_3 s^3}{s(\alpha_0 + \alpha_1 s + s^2)}
            \end{split}
        \end{equation*}\vspace{1em}
        
       \begin{quote}
        
        haben wir den folgenden Regler entworfen:
        
        \TODO{Regler}
        
        
    \end{quote}
    
    
    
    \subsection{Smith-Prädiktor}
    Aufgabe:\\
    Entwerfen Sie für die totzeitbehaftete Strecke einen Smith-Prädiktor unter Verwendung des Reglers aus
    \ref{aufg:3.1}\vspace{1em}
    
    
    \begin{quote}
        
        Der Smith-Prädiktor für die totzeitbehaftete Strecke einen unter Verwendung des Reglers aus \ref{aufg:3.1} ergab sich wie folgt:
        
        \TODO{scicos-Ding}
        
    \end{quote}
    
\end{quote} %Ende Section Vorbereitungsaufgaben 

%--------------------------------------------------------------------
%--------------------------------------------------------------------


\section{Simulation}
\begin{quote}
    
    \subsection{Stabilität mit steigender Totzeit}
    Aufgabe:\\
    Simulieren Sie den PID Regler zunächst ohne Totzeit mit einem Führungssprung der Amplitude $+30^{\circ}C$ mit dem
    idealen $PT2$-Modell! Fügen Sie dem Modell solange größer werdende Totzeiten ($T_d = 0.4, 0.8, . . .$) hinzu, bis
    der Regelkreis instabil wird! Beschreiben Sie kurz, welchen Einfluss die Totzeit auf das Regelkreisverhalten
    hat!\vspace{1em}
    
    \begin{quote}
    
    \end{quote}
    
    \subsection{Regelkreis mit PID-Regler}
    Aufgabe:\\
    Simulieren Sie den Regelkreis mit PID-Regler (ab hier immer mit Totzeit und dem vorgegebenen Scicos- Modell von der
    Webseite) unter einer sprungförmigen Referenz der Amplitude $+30^{\circ}C$ ! Kommentieren Sie kurz Ihre
    Beobachtungen!\vspace{1em}
    
    \begin{quote}
        
    \end{quote}
    
    
    \subsection{Smith-Prädiktor}
    Aufgabe:\\
    Implementieren Sie ihren Smith-Prädiktor und simulieren Sie erneut! Was ändert sich, was bleibt gleich?\vspace{1em}
    
    \begin{quote}
        
    \end{quote}
    
    
    \subsection{Pad\'e-Approximation}
    Aufgabe:\\
    Implementieren Sie nun auch den Regler auf Basis der Pad\'e-Approximation und simulieren Sie einen Führungssprung
    $+30^{\circ} C$ ! Wie ist die Regelgüte im Vergleich zum PID-Regler mit Smith-Prädiktor?\vspace{1em}
    
    \begin{quote}
        
    \end{quote}
    
    
    \subsection{Führungsverhalten beim Störfall}
    Aufgabe:\\
    Erproben sie nun einen Störfall: Erhöhen Sie die Totzeit (nur die des Modells, die Regler bleiben die gleichen)
    auf $T_d = 0.7s$ und simulieren Sie erneut das Führungsverhalten beider Regelkreise! Kommentieren Sie kurz das
    Ergebnis!\vspace{1em}
    
    \begin{quote}
        
    \end{quote}
    
    
    
\end{quote}

%--------------------------------------------------------------------
%--------------------------------------------------------------------


\section{Durchführung}
\begin{quote}
    
    
    \subsection{Arbeitspunkttemperatur}
    
    Aufgabe:\\
    Ermitteln Sie die Temperatur, die sich bei einer Heizleistung von $10$ einstellt und verwenden Sie diese als
    Arbeitspunkttemperatur!\vspace{1em}
    
    \begin{quote}
        
    \end{quote}
    
    
    \subsection{Führungssprung des PID-Reglers}
    
    Augabe:\\
    Implementieren Sie den PID-Regler (zunächst ohne Smith-Prädiktor) am Versuchsstand und führen Sie einen
    Führungssprung um $+30^{\circ} C$ aus! Die Stellgrößen als auch die Temperatur sind jeweils für jeden Versuch
    aufzuzeichen.\vspace{1em}
    
    \begin{quote}
        
    \end{quote}
    
    
    \subsection{Führungssprung mit Smith-Prädiktor}
    
    Aufgabe:\\
    Fügen Sie nun den Smith-Prädiktor hinzu und starten Sie wiederum das Experiment!\vspace{1em}
    
    \begin{quote}
        
    \end{quote}
    
    
    \subsection{Regler auf Approximationsbasis}
    
    Aufgabe:\\
    Erproben Sie ebenfalls den Regler auf Approximationsbasis!\vspace{1em}
    
    \begin{quote}
        
    \end{quote}
    
    
    \subsection{Vergleich der Messergebnisse}
    
    Aufgabe:\\
    Vergleichen Sie ihre Messergebnisse untereinander und mit den Simulationen!
    
    \begin{quote}
        
    \end{quote}
    
    
    
    
\end{quote}

%--------------------------------------------------------------------
%--------------------------------------------------------------------


\section{Auswertung}
\begin{quote}
    
\end{quote} %Ende Section Ergebnisse

%--------------------------------------------------------------------
%--------------------------------------------------------------------


\section{Scilabcode}
\begin{quote}
%     \lstinputlisting[
%         caption={Scilab-script},
%         language=scilab,
%         label=lst:scilab]
%         {./Scilab/Pendel2a.sce}


\end{quote} %Ende section

%--------------------------------------------------------------------
%--------------------------------------------------------------------    


%\begin{thebibliography}{999}
%\bibitem {Ueberschwingweite} Prof. Dr.-Ing. Raisch, Jörg; Dipl.-Ing. Hess, Anne-Katrin; Dipl.-Ing. Seel, Thomas:
%Grundlagen der Regelungstechnik - 4.Praktikum, S.5
%\bibitem {Ausregelzeit} Prof. Dr.-Ing. Raisch, Jörg; Dipl.-Ing. Hess, Anne-Katrin; Dipl.-Ing. Seel, Thomas:
%Grundlagen der Regelungstechnik - 4.Praktikum, S.5

%\usepackage{url}


%\bibitem{krachler}Christian Krachler:
%\href{http://www.krachler.com/fileadmin/user_upload/arbeiten/Reglersynthese_Christian_Krachler.pdf}{Reglersynthese nach
% dem Frequenzkennlinienverfahren}, S16, S22, 08.05.2012

%http://krachler.com/fileadmin/user\_upload/arbeiten/Reglersynthese\_Christian\_Krachler.pdf


%Name, Vorname.; evtl. Name2, Vorname2.: Titel des Dokumentes
%oder Buches, Zeitschrift/Verlag/URL (Auflage, Erscheinungsort, -jahr), ggf. Seitenzahlen
%\bibitem [Wiki10] {DigitaleMesskette2} \url{www.wikipedia.org}, Zugriff 22.03.2010
%\end{thebibliography}


\end{document}
