\newcommand{\institut}{}
\newcommand{\fachgebiet}{Regelungstechnik}
\newcommand{\veranstaltung}{Praktikum Grundlagen der Regelungstechnik}
\newcommand{\pdfautor}{Dirk Barbendererde (321 836), Boris Henckell (325 779)}
\newcommand{\autor}{Dirk Barbendererde (321 836)\\ Boris Henckell (325 779)}
\newcommand{\pdftitle}{Praktikum\ Regelungstechnik\ Versuch\ 3}
\newcommand{\prototitle}{Praktikum Regelungstechnik \\ Versuch 3}
\newcommand{\aufgabe}{}

\newcommand{\gruppe}{Gruppe: G1 Di 12-14}
\newcommand{\betreuer}{Betreuer: Markus Valtin}



\input{../../packages/tu_header_9}

\setcaptionwidth{7.5cm}

\begin{document}


%     \lstinputlisting{./praktikum6.sce}



%---------------------------------------------------------------------
%---------------------------------------------------------------------
%---------------------------------------------------------------------

\section{Reglerentwurf}
\begin{quote}
	\hspace{-2em}
	\subsection{PID-Regler}
	\label{aufg:3.1}
	
	Aufgabe:\\
	Entwerfen Sie für das totzeitfreie System
	
	\begin{equation*}
    	\begin{split}
    		\tilde{G}(s) = \frac{V_0}{(s-s_{\infty 1})(s-s_{\infty 2})}
    	\end{split}
    \end{equation*}
	
	einen (realen) PID-Regler $K_{PID}$, indem sie beide Polstellen der Strecke kürzen. Die übrigen Parameter sollen so
	bestimmt werden, dass die Dynamik des Führungsverhaltens des resultierenden Regelkreises mit dem des folgenden Polpaares übereinstimmt:
	
	\begin{equation*}
    	\begin{split}
    		P(s) = \frac{1}{s^2 \omega^2 + s \frac{2 d}{\omega} + 1}
    	\end{split}
    \end{equation*}
	
	Der Regelkreis soll also relativ schnell sein, jedoch ohne dass dabei Überschwingen auftritt.\vspace{3em}
	
    \begin{quote}
        
        Zu Beginn des Reglerentwurfs definieren wir die uns vorgegebene linearisierte Stecke $\tilde{G}$.\\
        Anschließend erstellen wir folgendermaßen unseren PID-Regler:
        \begin{equation*}
        	\begin{split}
        		\tilde{K} &= K_{PID} \  \frac{1}{1 + T s} \ \frac{1}{s} (s^2  T_d T_i + T_i s + 1)\\
        	\end{split}
        \end{equation*}
        
        mit 
        
        \begin{equation*}
        	\begin{split}
        		T_i &= \frac{-(s_1 + s_2)}{s_1 s_2}\\
        		T_d &= \frac{1}{-(s_1 + s_2)}\\
        	\end{split}
        \end{equation*}
        
        In diesem Fall stellen die beiden Variablen $s_1$ und $s_2$ die Polstellen der Strecke $\tilde{G}$ dar. Auf
        diese Weise kürzen wir diese schon mal durch den Regeler. Verbleiben noch die Position des Realisierbarkeitspols
        sowie die Verstärkung des Reglers als Parameter um die Dynamik des Führungsverhaltens des Reglers zu
        beeinflussen.\\
        
        Als Vergleich erstellen wir die Führungssprungantwort des gegebenen Polpaars.\\
        
        Mit einer Versärkung von $K_{PID} = 1.6$ und einem Realisierbarkeitspol bei $-20$ erhalten wir folgende
        Führungssprünge:
        
        \begin{figure}[H]
        \centering
            \includegraphics[scale=0.7, trim = 0cm 0cm 0cm 0cm, clip]{./Bilder/Sprungantwort}
                \caption{Führungssprungantwort}
        \end{figure}
    
    \end{quote}
    
    
    \subsection{Pad\'e-Approximation}
    
    Aufgabe:\\
    Approximieren Sie den Term $e^{-sT_d}$, welcher für die Totzeit verantwortlich ist, mithilfe einer
    Pad\'e-Approximation erster Ordnung durch eine gebrochen-rationale Funktion und stellen Sie das approximierte
    Streckenmodell $\tilde{G}$ auf, indem Sie die Totzeit in $G$ durch ihre Approximation ersetzen. \vspace{1em}
    
    \begin{quote}
        
        Als nächstes haben wir die Totzeit mit einer Pad\'e-Approximation erster Ordnung durch eine gebrochen-rationale
        Funktion ersetzt und multiplizieren sie an unsere Strecke. Für diese Näherung des Systems ($\hat{G}$) entwerfen
        wir nun einen Regler.\vspace{1em}
        
        Die Approximation lautet:
        \begin{equation*}
        	\begin{split}
        		p_1(s) = \frac{1-\frac{\tau}{2}s}{1+\frac{\tau}{2}s}
        	\end{split}
        \end{equation*}
        
    \end{quote}
    
    \subsection{Reglerentwurf für die Strecke $\hat{G}$}
    Aufgabe:\\
    Es soll ein Regler für das approximierte Streckenmodell $\hat{G}$ mittels Polvorgabe entworfen werden, der
    folgende Eigenschaften des geschlossenen Kreises ermöglicht:
        
        
        \begin{itemize}
            
            \item kein Überschwingen der Regelgröße bei sprungförmigen Führungs- oder Störsignalen
            
            \item ungefähr gleiche Anstiegszeit der Regelgröße wie bei der Verwendung des PID-Regler aus \ref{aufg:3.1}
            
            \item Regelfehler $\to$ 0 für t $\to$ $\infty$ unter sprungförmigen Referenzen
        
        \end{itemize}
        \vspace{1em}
        
        \begin{enumerate}
            
            \item Stellen Sie den Regleransatz mit kleinstmöglicher Nennerordnung und Integratoranteil auf. Wie viele Pole
            müssen Sie vorgeben?
            
            \item Stellen Sie das Sollpolpolynom auf; verwenden sie hierzu die Pole des Polpaares aus Aufgabe \ref{aufg:3.1}
            und die Pole der Strecke $G$, wählen sie einen weiteren Pol bei ($s_\infty = -2$).
            
            \item Stellen Sie die Sylvester Matrix durch Koeffizientenvergleich des Polpolynoms des geschlossenen Kreises
            mit Ihrem Sollpolpolynom auf und berechnen sie die Reglerparameter mithilfe von Scilab.
        
        \end{enumerate}\vspace{1em}
       
        Mit dem Regleransatz\\
        
        \begin{equation*}
            \begin{split}
                K(s) = \frac{\beta_0 + \beta_1 s + \beta_2 s^2 + \beta_3 s^3}{s(\alpha_0 + \alpha_1 s + s^2)}
            \end{split}
        \end{equation*}\vspace{1em}
        
    \begin{quote}
      
        Das approximierte Streckenmodell $\hat{G}$ allein hat die Ordnung $n = 3$. Da der Regler einen
        Integratoranteil besitzen soll und wir den Reglerentwurf wählen bei dem dieser Integrator im nachhinein
        an einen Regler multipliziert wird, müssen wir die Strecke anpassen und erhalten eine neue Streckenordnung von
        $n = 4$. Daraus folgt, dass die zu berechnende Sylvestermatrix $8\cdot 8$ Einträge und das Sollpolynom 8
        Einträge besitzen muss. Um das zu realisieren müssen wir insgesamt 7 Pole vorgeben, da der letzte Eintrag des
        Sollpolynoms der Koeffizent vor dem $s^0$ ist.\\
        
        Neben dem Pol bei $s_\infty = -2$ und den zwei Polen des Polpaares benötigen wir noch genau 4 Pole aus der
        veränderten Strecke. Unsere Strecke ergibt sich aus:
        
        \begin{equation*}
        	\begin{split}
        		\hat{G_I} = \hat{G} \frac{1}{s}
        	\end{split}
        \end{equation*}
        
        und hat genau 4 Pole.\\
        
        Insgesamt ergibt sich daraus das folgende Sollpolpolynom:
        
        \begin{equation*}
        	\begin{split}
        		q_{soll} = 32.148065s + 177.77307s^2 + 285.03595s^3 + 208.51884s^4 + 77.572633s^5 + 14.179911s^6 + s^7  
        	\end{split}
        \end{equation*}
        \vspace{1em}
        
        Anschließend haben wir die Sylvestermatrix aufgestellt und mithilfe des Sollpolpolynoms die Koeffizenten des
        Reglers errechnet.\\
        Unser Regler sieht folgendermaßen aus:
        
        \begin{equation*}
        	\begin{split}
        		KposI = \frac{14.231104 +  56.480246s +  26.455258s^2 +  3.1456906s^3}{62.570515s +  28.849556s^2 +  5s^3}
        	\end{split}
        \end{equation*}
        
        Damit entspricht er dem geforderten Regleransatz.\\
        Die Kontrolle der Polstellen des resultierenden geschlossenen Regelkreises ergibt, dass sie mit denen des
        Sollpolpolynoms übereinstimmen.
    \end{quote}
    
    
    
    \subsection{Smith-Prädiktor}
    Aufgabe:\\
    Entwerfen Sie für die totzeitbehaftete Strecke einen Smith-Prädiktor unter Verwendung des Reglers aus
    \ref{aufg:3.1}\vspace{1em}
    
    
    \begin{quote}
        
        Der Smith-Prädiktor für die totzeitbehaftete Strecke unter Verwendung des Reglers aus \ref{aufg:3.1} ergab sich wie folgt:
        
       \begin{figure}[H]
        \centering
            \includegraphics[scale=0.4, trim = 0cm 0cm 0cm 0cm, clip]{./Bilder/Smith-Praediktor}
                \caption{Smith-Prädiktor}
        \end{figure}
    
        
    \end{quote}
    
\end{quote} %Ende Section Reglerentwurf 

%--------------------------------------------------------------------
%--------------------------------------------------------------------


\section{Simulation}
\begin{quote}
    
    
    
    \subsection{Stabilität mit steigender Totzeit}
    \begin{quote}
        
    \end{quote}
    Aufgabe:\\
    Simulieren Sie den PID Regler zunächst ohne Totzeit mit einem Führungssprung der Amplitude $+30^{\circ}C$ mit dem
    idealen $PT2$-Modell! Fügen Sie dem Modell solange größer werdende Totzeiten ($T_d = 0.4, 0.8, . . .$) hinzu, bis
    der Regelkreis instabil wird! Beschreiben Sie kurz, welchen Einfluss die Totzeit auf das Regelkreisverhalten
    hat!\vspace{1em}
            
        \begin{figure}[H]
        \centering
            \includegraphics[scale=0.4, trim = 0cm 0cm 0cm 0cm, clip]{./Bilder/4_1_Simulation}
                \caption{Simulationsergebnisse}
        \end{figure}
        
        Anhand der fünf Simulationen mit steigender Totzeit lässt sich erkennen, dass die Sprungantwort zunehmend
        überschwingt. Bei dieser Simulation weiß der Regler nichts von der Totzeit und erwartet eine sofortige
        Reaktion auf seine Reglung. Da diese jedoch erst verspätet beim Regler ankommt übersteuert er. Je größer die
        Totzeit ist umso mehr fällt dieser Effekt auf. In unserem Fall schafft es der Regler für eine Totzeit von $1.6
        \ s$ nicht mehr, aus dem Überwingen rauszukommen und wir instabil.\\
        \vspace{1em}
    
    \end{quote}
    
    \vspace{3em}
    
    Die folgenden 5 Simulationen sind im nächsten Bild zusammengefasst und werden nachfolgend betrachtet.
    
    \begin{figure}[H]
    \centering
        \includegraphics[scale=0.4, trim = 0cm 0cm 0cm 0cm, clip]{./Bilder/4_2_Simulation_rest}
            \caption{Simulation rest}
    \end{figure}
    
        
    \subsection{Regelkreis mit PID-Regler}
    Aufgabe:\\
    Simulieren Sie den Regelkreis mit PID-Regler (ab hier immer mit Totzeit und dem vorgegebenen Scicos- Modell von der
    Webseite) unter einer sprungförmigen Referenz der Amplitude $+30^{\circ}C$ ! Kommentieren Sie kurz Ihre
    Beobachtungen!\vspace{1em}
    
        
        Bei dieser Simulation wurde anstatt des linearisierten Modells der Stecke, ein nichtlinearisiertes Modell
        verwandt. Es fällt auf, dass bei diesem Modell die Temperatur des Arbeitspunktes von Anfang an konstant ist. Die
        Sprungantwort reagiert mit auffallendem Überschwingen als das lineare Modelle mit der selben
        Totzeit. Das liegt unserer Meinung nach an dem genaueren Modell der Strecke. Bisher wurde noch kein Versuch
        unternommen die Totzeit zu kompensieren\\
        \vspace{1em}
        
        
    \end{quote}
    
    \subsection{Smith-Prädiktor}
    Aufgabe:\\
    Implementieren Sie ihren Smith-Prädiktor und simulieren Sie erneut! Was ändert sich, was bleibt gleich?\vspace{1em}
    
        
        Bei dieser Simulation wurde ein Smith-Prädiktor hinzugefügt. Es ist ein auffallender Unterschied zu erkenne.
        zwar schwingt die Antwort noch ein wenig über jedoch bei weitem nicht soviel wie ohne Smith-Prädiktor.
        \vspace{1em}
        
    \end{quote}
    
    
    \subsection{Pad\'e-Approximation}
    Aufgabe:\\
    Implementieren Sie nun auch den Regler auf Basis der Pad\'e-Approximation und simulieren Sie einen Führungssprung
    $+30^{\circ} C$ ! Wie ist die Regelgüte im Vergleich zum PID-Regler mit Smith-Prädiktor?\vspace{1em}
   
        
        Bei dem Regler mit Pad\'e Approximation verschlechtert sich das Sprungverhalten wieder. Das Überschwingen nimmt
        zu und auch die reaktion verlangsamt sich. Das ist auch nicht verwunderlich, da unter anderem die
        Totzeit bei diesem Regeler nur angenähert wurde, während die Totzeit beim Smith-Prädiktor durch ein ``Delay''-Block dargestellt wurde, der genauer arbeitet als die Approximation.\\
        \vspace{1em}
        
    \end{quote}
    
    
    \subsection{Führungsverhalten beim Störfall}
    Aufgabe:\\
    Erproben sie nun einen Störfall: Erhöhen Sie die Totzeit (nur die des Modells, die Regler bleiben die gleichen)
    auf $T_d = 0.7s$ und simulieren Sie erneut das Führungsverhalten beider Regelkreise! Kommentieren Sie kurz das
    Ergebnis!\vspace{1em}
    
    \begin{quote}
    
        
        Bei der verwendeten Störung sind beide Sprungantworten wieder von einem Starken überschwingen geprägt. Jedoch
        schwingt der Regler mit Smith-Prädiktor weniger über,
        ist also deutlich stailer.
        Wir erwarten daher, dass er in jeder hinsicht besser ist als der Regler mit Pad\'e-Approximation.\\
        \vspace{1em}
        
    \end{quote}
    
    
    
\end{quote}

%--------------------------------------------------------------------
%--------------------------------------------------------------------


\section{Durchführung}
\begin{quote}
    
    
    \subsection{Arbeitspunkttemperatur}
    \begin{quote}
        
    \end{quote}
    Aufgabe:\\
    Ermitteln Sie die Temperatur, die sich bei einer Heizleistung von $10$ einstellt und verwenden Sie diese als
    Arbeitspunkttemperatur!\vspace{1em}
    
    \begin{quote}
        
        Der Arbeitspunkt mit einer Heizleistung von 10 lag bei $31^{\circ} C$. Da die Außentemperatur sehr hoch war hat
        sich das System jedoch nicht immer wieder vollständig auf den Arbeitspunkt abgekühlt, weshalb einige Messungen
        bei etwas über dem Arbeitspunkt starten.\\
        
        
    \end{quote}
    
    
    
    \subsection{Führungssprung}
    
    Augabe:\\
    Implementieren Sie den PID-Regler (zunächst ohne Smith-Prädiktor) am Versuchsstand und führen Sie einen
    Führungssprung um $+30^{\circ} C$ aus! Die Stellgrößen als auch die Temperatur sind jeweils für jeden Versuch
    aufzuzeichen.\\
    Fügen Sie nun den Smith-Prädiktor hinzu und starten Sie wiederum das Experiment!\\
    Erproben Sie ebenfalls den Regler auf Approximationsbasis!\vspace{1em}
    
    
    
    \begin{quote}
        Nun haben wir alle drei Regler nacheinander implementiert und getestet. Wir haben, um die Führungssprungantwort zu
        messen, einen Sprung von $31^{\circ} C$ nach $61^{\circ} C$ auf die Regelstrecke gegeben.\\
        
        \begin{figure}[H]
            \centering
            \includegraphics[scale=1, trim = 1cm 0.5cm 1cm 1cm, clip]{Bilder/5}
            \caption{Gemessene Führungssprungantwort}
            \label{fig:5}
        \end{figure}
        \vspace{1em}
        
        Unerwarteter Weise verhielt sich das System überhaupt nicht wie Simuliert. Nähere Nachforschungen ergaben, dass
        sich die Parameter der echten Strecke zu stark von denen des vorgegeben Modells unterscheiden hatten. Die
        Einstellungen des Systems waren beim Versuchsaufbau also deutlich anders als wir beim Entwurf der Regler
        angenommen hatten.\\
        Vor allem war die Streckenverstärkung deutlich geringer als wir annahmen, weshalb die Verstärkung
        unserer Regler nicht ausreichend war. In Abb. \ref{fig:5} ist deutlich zu erkennen, dass es alle
        unsere Regler nach 15 Sekunden gerade einmal geschafft haben auf $40^{\circ}$ zu heizen obwohl sie schon nach
        ca. 5 Sekunden bei $60^{\circ}$ hätten sein sollen.\\
        Somit sind die Messergebnisse offensichtlich nicht mit den Simulierten vergleichbar. Aufgrund des falschen
        Reglerentwurfs sind leider auch keine rückschlüsse auf die Eignung bzw. Qualität der Regler möglich.
        
        
        
        
    \end{quote}
    
    
    
    
\end{quote}

%--------------------------------------------------------------------
%--------------------------------------------------------------------


\section{Auswertung}
\begin{quote}
    
    \subsection{Auswirkungen von Totzeiten}
    \begin{quote}
        
        Anhand der Simulationen haben wir beispielhaft herausgefunden wie sich Totzeiten auf ein Regelsystem
        auswirken.\\
        Falls die Totzeit eines Systems beim Reglerentwurf nicht berücksichtigt wird kommt es, je nach länge der
        Verzögerung, zu Überschwingen der Sprungantwort. Genügend große Totzeiten können den Regler sogar instabil
        werden laseen, falls der Regler während der Totzeit so doll weiter regelt, dass der Integrator über die
        Begrenzung hinweg voll läuft und so eine art Windup-Effekt auftritt.\\ 
        
    \end{quote} % Sub: Auswirkungen von Totzeiten
    
    
    \subsection{Reglerentwurf}
    \begin{quote}
        
        Für den Reglerentwurf bei Strecken mit Totzeit hat sich in unseren Simulationen der Smith-Prädiktor
        durchgesetzt. Sowohl das Führungssprung- als auch das Störverhalten waren besser als beim Regler mit Pad\'e
        Approximation. Hierbei spielt es wahrscheinlich auch eine Rolle, dass bei letzterem für die Totzeit nur eine
        Näherung verwand wird, wohingegen der Smith-Prädiktor eine genauere Verzögerungs\-komponente hat.\\
        Schlussendlich hat der Smith-Prädiktor zwar eine höhere Reglergüte, der Reglerentwurf mittels einer
        Approximierten Totzeit ist aber einfacher und mit weniger Aufwand um zu setzen..\\
        
    \end{quote} % Sub: Reglerentwurf
    \vspace{1em}
    
    
    
\end{quote} %Ende Section Ergebnisse

%--------------------------------------------------------------------
%--------------------------------------------------------------------


\section{Scilabcode}
\begin{quote}
    \lstinputlisting[
        caption={Scilab-script},
        language=scilab,
        label=lst:scilab]
        {./Scilab/Luft-Temperatur-Regelstrecke.sce}


\end{quote} %Ende section

%--------------------------------------------------------------------
%--------------------------------------------------------------------    


%\begin{thebibliography}{999}
%\bibitem {Ueberschwingweite} Prof. Dr.-Ing. Raisch, Jörg; Dipl.-Ing. Hess, Anne-Katrin; Dipl.-Ing. Seel, Thomas:
%Grundlagen der Regelungstechnik - 4.Praktikum, S.5
%\bibitem {Ausregelzeit} Prof. Dr.-Ing. Raisch, Jörg; Dipl.-Ing. Hess, Anne-Katrin; Dipl.-Ing. Seel, Thomas:
%Grundlagen der Regelungstechnik - 4.Praktikum, S.5

%\usepackage{url}


%\bibitem{krachler}Christian Krachler:
%\href{http://www.krachler.com/fileadmin/user_upload/arbeiten/Reglersynthese_Christian_Krachler.pdf}{Reglersynthese nach
% dem Frequenzkennlinienverfahren}, S16, S22, 08.05.2012

%http://krachler.com/fileadmin/user\_upload/arbeiten/Reglersynthese\_Christian\_Krachler.pdf


%Name, Vorname.; evtl. Name2, Vorname2.: Titel des Dokumentes
%oder Buches, Zeitschrift/Verlag/URL (Auflage, Erscheinungsort, -jahr), ggf. Seitenzahlen
%\bibitem [Wiki10] {DigitaleMesskette2} \url{www.wikipedia.org}, Zugriff 22.03.2010
%\end{thebibliography}


\end{document}
