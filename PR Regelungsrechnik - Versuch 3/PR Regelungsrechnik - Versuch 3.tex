\newcommand{\institut}{}
\newcommand{\fachgebiet}{Regelungstechnik}
\newcommand{\veranstaltung}{Praktikum Grundlagen der Regelungstechnik}
\newcommand{\pdfautor}{Dirk Barbendererde (321 836), Boris Henckell (325 779)}
\newcommand{\autor}{Dirk Barbendererde (321 836)\\ Boris Henckell (325 779)}
\newcommand{\pdftitle}{Praktikum\ Regelungstechnik\ Versuch\ 3}
\newcommand{\prototitle}{Praktikum Regelungstechnik \\ Versuch 3}
\newcommand{\aufgabe}{}

\newcommand{\gruppe}{Gruppe: G1 Di 12-14}
\newcommand{\betreuer}{Betreuer: Markus Valtin}



\input{../../packages/tu_header_9}
%\begin{document}


%     \lstinputlisting{./praktikum6.sce}



%---------------------------------------------------------------------
%---------------------------------------------------------------------
%---------------------------------------------------------------------

\section{Reglerentwurf}
\begin{quote}
	\hspace{-2em}
	\subsection{PID-Regler}
	\label{aufg:3.1}
    \begin{quote}
        
        Da wir für ein nichtlineares System mit Totzeit keinen Regler entwerfen können entwerfen wir zunächst einen
        Regler für das folgende totzeitfreie System:
        
        \TODO{Boris: Formel}
        
        Die Dynamik des Führungsverhaltens des resultierenden Regelkreises soll mit dem des folgenden Polpaares
        ubereinstimmt:
        
        \TODO{Boris: Fornell}
        
        
    \end{quote}
    
    
    \subsection{Pad\'e-Approximation}
    \begin{quote}
        
        Als nächstes haben wir die Totzeit mit einer Pad\'e-Approximation erster Ordnung durch eine gebrochen-rationale
        Funktion ersetzt und multiplizieren sie an unsere Strecke. Für diese Näherung des Systems ($\hat{G}$) entwerfen
        wir nun einen Regler.
        
        
    \end{quote}
    
    \subsection{Reglerentwurf für die Strecke $\hat{G}$}
    \begin{quote}
        
        Mit dem Regleransatz\\
        
        \TODO{Boris: Farmvill}
        
        und den Vorgeben\\
        
        \begin{itemize}
            
            \item kein Uberschwingen der Regelgröße bei sprungförmigen Führungs- oder Störsignalen
            
            \item ungefähr gleiche Anstiegszeit der Regelgröße wie bei der Verwendung des PID-Regler aus
            \ref{aufg:3.1}
            
            \item Regelfehler $\to$ 0 für t $\to$ $\infty$ unter sprungförmigen Referenzen
        
        \end{itemize}
        \vspace{1em}
        
        \begin{enumerate}
            
            \item Stellen Sie den Regleransatz mit kleinstmöglicher Nennerordnung und Integratoranteil auf. Wie viele Pole
            müssen Sie vorgeben?\\
            
            \item Stellen Sie das Sollpolpolynom auf; verwenden sie hierzu die Pole des Polpaares aus Aufgabe \ref{aufg:3.1}
            und die Pole der Strecke G, wählen sie einen weiteren Pol bei ($s_\infty = -2$).\\
            
            \item Stellen Sie die Sylvester Matrix durch Koeffizientenvergleich des Polpolynoms des geschlossenen Kreises
            mit Ihrem Sollpolpolynom auf und berechnen sie die Reglerparameter mithilfe von Scilab.\\
        
        \end{enumerate}
        
        
        haben wir den folgenden Regler entworfen:
        
        \TODO{Regler}
        
        
    \end{quote}
    
    
    
    \subsection{Smith-Prädiktor}
    \begin{quote}
        
        Der Smith-Prädiktor für die totzeitbehaftete Strecke einen unter Verwendung des Reglers aus \ref{aufg:3.1} ergab sich wie folgt:
        
        \TODO{scicos-Ding}
        
    \end{quote}
    
\end{quote} %Ende Section Vorbereitungsaufgaben 

%--------------------------------------------------------------------
%--------------------------------------------------------------------


\section{Simulation}
\begin{quote}
    
    \subsection{Stabilität mit steigender Totzeit}
    Simulieren Sie den PID Regler zunächst ohne Totzeit mit einem Führungssprung der Amplitude
    \SI{+30}{\celsius} mit dem idealen PT2-Modell! Fügen Sie dem Modell solange größer werdende Totzeiten (Td = 0.4,
    0.8, \ldots) hinzu, bis der Regelkreis instabil wird! Beschreiben Sie kurz, welchen Einfluss die Totzeit auf das
    Regelkreisverhalten hat!
    \begin{quote}
        
    \end{quote}
    
    
    \subsection{Regelkreis mit PID-Regler}
    \begin{quote}
        
    \end{quote}
    
    
    \subsection{Smith-Prädiktor}
    \begin{quote}
        
    \end{quote}
    
    
    \subsection{Pad\'e-Approximation}
    \begin{quote}
        
    \end{quote}
    
    
    \subsection{Führungsverhalten beim Störfall}
    \begin{quote}
        
    \end{quote}
    
    
    
\end{quote}

%--------------------------------------------------------------------
%--------------------------------------------------------------------


\section{Durchführung}
\begin{quote}
    
    
    \subsection{Arbeitspunkttemperatur}
    \begin{quote}
        
    \end{quote}
    
    
    \subsection{Führungssprung des PID-Reglers}
    \begin{quote}
        
    \end{quote}
    
    
    \subsection{Führungssprung mit Smith-Prädiktor}
    \begin{quote}
        
    \end{quote}
    
    
    \subsection{Regler auf Approximationsbasis}
    \begin{quote}
        
    \end{quote}
    
    
    \subsection{Vergleich der Messergebnisse}
    \begin{quote}
        
    \end{quote}
    
    
    
    
\end{quote}

%--------------------------------------------------------------------
%--------------------------------------------------------------------


\section{Auswertung}
\begin{quote}
    
\end{quote} %Ende Section Ergebnisse

%--------------------------------------------------------------------
%--------------------------------------------------------------------


\section{Scilabcode}
\begin{quote}
%     \lstinputlisting[
%         caption={Scilab-script},
%         language=scilab,
%         label=lst:scilab]
%         {./Scilab/Pendel2a.sce}


\end{quote} %Ende section

%--------------------------------------------------------------------
%--------------------------------------------------------------------    


%\begin{thebibliography}{999}
%\bibitem {Ueberschwingweite} Prof. Dr.-Ing. Raisch, Jörg; Dipl.-Ing. Hess, Anne-Katrin; Dipl.-Ing. Seel, Thomas:
%Grundlagen der Regelungstechnik - 4.Praktikum, S.5
%\bibitem {Ausregelzeit} Prof. Dr.-Ing. Raisch, Jörg; Dipl.-Ing. Hess, Anne-Katrin; Dipl.-Ing. Seel, Thomas:
%Grundlagen der Regelungstechnik - 4.Praktikum, S.5

%\usepackage{url}


%\bibitem{krachler}Christian Krachler:
%\href{http://www.krachler.com/fileadmin/user_upload/arbeiten/Reglersynthese_Christian_Krachler.pdf}{Reglersynthese nach
% dem Frequenzkennlinienverfahren}, S16, S22, 08.05.2012

%http://krachler.com/fileadmin/user\_upload/arbeiten/Reglersynthese\_Christian\_Krachler.pdf


%Name, Vorname.; evtl. Name2, Vorname2.: Titel des Dokumentes
%oder Buches, Zeitschrift/Verlag/URL (Auflage, Erscheinungsort, -jahr), ggf. Seitenzahlen
%\bibitem [Wiki10] {DigitaleMesskette2} \url{www.wikipedia.org}, Zugriff 22.03.2010
%\end{thebibliography}


\end{document}
